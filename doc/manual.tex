\documentclass[a4paper]{article}

% Linguagem
\usepackage[utf8]{inputenc}
\usepackage[portuguese]{babel}
\usepackage[T1]{fontenc}

% Pacotes matemáticos
\usepackage{amsmath}
\usepackage{amsfonts}
\usepackage{amssymb}
\usepackage{graphicx}

% Fontes e identaçãp
\usepackage{setspace}                   % espaçamento flexível
\usepackage{indentfirst}                % indentação do primeiro parágrafo
\usepackage[fixlanguage]{babelbib}
\usepackage[font=small,format=plain,labelfont=bf,up,textfont=it,up]{caption}

% Pacotes para cores e modelos
\usepackage[a4paper,top=3.0cm,bottom=2.0cm,left=3.0cm,right=2.0cm]{geometry} \usepackage[usenames,svgnames,dvipsnames]{xcolor}
\usepackage[pdftex,plainpages=false,pdfpagelabels,pagebackref,
            colorlinks=true,citecolor=DarkGreen,linkcolor=DarkRed,
            urlcolor=DarkRed,filecolor=DarkGreen,
            bookmarksopen=true]{hyperref}

% Pacotes para itens
\usepackage{calc}  
\usepackage{enumitem}  

\title  {Projeto de Laboratório de Programação II - Fase 2}
\author {Karina Suemi, Vinícius Silva, Renato Cordeiro}
\date   {}

%%%%%%%%%%%%%%%%%%%%%%%%%%%%%%%%%%%%%%%%%%%%%%%%%%%%%%%%%%%%%%%%%%%%%%%%

\begin{document}

\maketitle

\bigskip

{\textcolor{NavyBlue}{\Huge ..Manual do Usuário..}

\newpage

{\textcolor{NavyBlue}{\LARGE INSTALAÇÃO }
                
\bigskip
                            
Para compilar o jogo, digite:    
                                            
\$ \textit {ant}                                   

\bigskip
\bigskip
\bigskip

{\textcolor{NavyBlue}{\LARGE JOGO }

O  jogo consiste  em programar uma  série de 
robôs para batalharem, num estilo de RTS 2x2.
Para  tanto,  os robôs devem  ter suas ações 
programadas. Eles irão executá-las até que o 
jogo acabe ou sejam destruídos.

Nesta fase do desenvolvimento, a programação 
deve  ser  feita   em  linguagem  *Assembly*,
desenvolvida  especialmente  para  a máquina 
virtual  em *Java*. 

Os programas devem  ser criados com extensão 
*.asm*.   Exemplos   estão   disponíveis  no 
diretório `test/` junto ao código-fonte. 

Para  utilizá-los  como  programas  para  os 
robôs, compile-os com:

\$ \textit {sh reload.sh path/para/o/arquivo.asm}

\bigskip
                                            
E para iniciar o jogo: 
                                            
\$ \textit{ java -jar dist/MAC0242-Project.jar programa\_jogador\_1 programa\_jogador\_2 } 

\bigskip

Caso queira executar o programa em modo 
Debugger, digite:

\$ \textit {java -jar dist/MAC0242-Project.jar }
      programa\_jogador\_1 programa\_jogador\_2 -v

\bigskip

{\textcolor{NavyBlue}{\LARGE DOCUMENTAÇÃO }

A   documentação    do   código-fonte   está 
disponível no formato [Javadoc][4] e no
formato relatório Latex para compreendimento
do código.







\end{document}
